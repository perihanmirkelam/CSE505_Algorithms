
% Default to the notebook output style

    


% Inherit from the specified cell style.




    
\documentclass[11pt, fleqn]{article}

    
    
    \usepackage[T1]{fontenc}
    % Nicer default font (+ math font) than Computer Modern for most use cases
    \usepackage{mathpazo}

    % Basic figure setup, for now with no caption control since it's done
    % automatically by Pandoc (which extracts ![](path) syntax from Markdown).
    \usepackage{graphicx}
    % We will generate all images so they have a width \maxwidth. This means
    % that they will get their normal width if they fit onto the page, but
    % are scaled down if they would overflow the margins.
    \makeatletter
    \def\maxwidth{\ifdim\Gin@nat@width>\linewidth\linewidth
    \else\Gin@nat@width\fi}
    \makeatother
    \let\Oldincludegraphics\includegraphics
    % Set max figure width to be 80% of text width, for now hardcoded.
    \renewcommand{\includegraphics}[1]{\Oldincludegraphics[width=.8\maxwidth]{#1}}
    % Ensure that by default, figures have no caption (until we provide a
    % proper Figure object with a Caption API and a way to capture that
    % in the conversion process - todo).
    \usepackage{caption}
    \DeclareCaptionLabelFormat{nolabel}{}
    \captionsetup{labelformat=nolabel}

    \usepackage{adjustbox} % Used to constrain images to a maximum size 
    \usepackage{xcolor} % Allow colors to be defined
    \usepackage{enumerate} % Needed for markdown enumerations to work
    \usepackage{geometry} % Used to adjust the document margins
    \usepackage{amsmath} % Equations
    \usepackage{amssymb} % Equations
    \usepackage{textcomp} % defines textquotesingle
    % Hack from http://tex.stackexchange.com/a/47451/13684:
    \AtBeginDocument{%
        \def\PYZsq{\textquotesingle}% Upright quotes in Pygmentized code
    }
    \usepackage{upquote} % Upright quotes for verbatim code
    \usepackage{eurosym} % defines \euro
    \usepackage[mathletters]{ucs} % Extended unicode (utf-8) support
    \usepackage[utf8x]{inputenc} % Allow utf-8 characters in the tex document
    \usepackage{fancyvrb} % verbatim replacement that allows latex
    \usepackage{grffile} % extends the file name processing of package graphics 
                         % to support a larger range 
    % The hyperref package gives us a pdf with properly built
    % internal navigation ('pdf bookmarks' for the table of contents,
    % internal cross-reference links, web links for URLs, etc.)
    \usepackage{hyperref}
    \usepackage{longtable} % longtable support required by pandoc >1.10
    \usepackage{booktabs}  % table support for pandoc > 1.12.2
    \usepackage[inline]{enumitem} % IRkernel/repr support (it uses the enumerate* environment)
    \usepackage[normalem]{ulem} % ulem is needed to support strikethroughs (\sout)
                                % normalem makes italics be italics, not underlines
    \usepackage{mathrsfs}
    

    
    
    % Colors for the hyperref package
    \definecolor{urlcolor}{rgb}{0,.145,.698}
    \definecolor{linkcolor}{rgb}{.71,0.21,0.01}
    \definecolor{citecolor}{rgb}{.12,.54,.11}

    % ANSI colors
    \definecolor{ansi-black}{HTML}{3E424D}
    \definecolor{ansi-black-intense}{HTML}{282C36}
    \definecolor{ansi-red}{HTML}{E75C58}
    \definecolor{ansi-red-intense}{HTML}{B22B31}
    \definecolor{ansi-green}{HTML}{00A250}
    \definecolor{ansi-green-intense}{HTML}{007427}
    \definecolor{ansi-yellow}{HTML}{DDB62B}
    \definecolor{ansi-yellow-intense}{HTML}{B27D12}
    \definecolor{ansi-blue}{HTML}{208FFB}
    \definecolor{ansi-blue-intense}{HTML}{0065CA}
    \definecolor{ansi-magenta}{HTML}{D160C4}
    \definecolor{ansi-magenta-intense}{HTML}{A03196}
    \definecolor{ansi-cyan}{HTML}{60C6C8}
    \definecolor{ansi-cyan-intense}{HTML}{258F8F}
    \definecolor{ansi-white}{HTML}{C5C1B4}
    \definecolor{ansi-white-intense}{HTML}{A1A6B2}
    \definecolor{ansi-default-inverse-fg}{HTML}{FFFFFF}
    \definecolor{ansi-default-inverse-bg}{HTML}{000000}

    % commands and environments needed by pandoc snippets
    % extracted from the output of `pandoc -s`
    \providecommand{\tightlist}{%
      \setlength{\itemsep}{0pt}\setlength{\parskip}{0pt}}
    \DefineVerbatimEnvironment{Highlighting}{Verbatim}{commandchars=\\\{\}}
    % Add ',fontsize=\small' for more characters per line
    \newenvironment{Shaded}{}{}
    \newcommand{\KeywordTok}[1]{\textcolor[rgb]{0.00,0.44,0.13}{\textbf{{#1}}}}
    \newcommand{\DataTypeTok}[1]{\textcolor[rgb]{0.56,0.13,0.00}{{#1}}}
    \newcommand{\DecValTok}[1]{\textcolor[rgb]{0.25,0.63,0.44}{{#1}}}
    \newcommand{\BaseNTok}[1]{\textcolor[rgb]{0.25,0.63,0.44}{{#1}}}
    \newcommand{\FloatTok}[1]{\textcolor[rgb]{0.25,0.63,0.44}{{#1}}}
    \newcommand{\CharTok}[1]{\textcolor[rgb]{0.25,0.44,0.63}{{#1}}}
    \newcommand{\StringTok}[1]{\textcolor[rgb]{0.25,0.44,0.63}{{#1}}}
    \newcommand{\CommentTok}[1]{\textcolor[rgb]{0.38,0.63,0.69}{\textit{{#1}}}}
    \newcommand{\OtherTok}[1]{\textcolor[rgb]{0.00,0.44,0.13}{{#1}}}
    \newcommand{\AlertTok}[1]{\textcolor[rgb]{1.00,0.00,0.00}{\textbf{{#1}}}}
    \newcommand{\FunctionTok}[1]{\textcolor[rgb]{0.02,0.16,0.49}{{#1}}}
    \newcommand{\RegionMarkerTok}[1]{{#1}}
    \newcommand{\ErrorTok}[1]{\textcolor[rgb]{1.00,0.00,0.00}{\textbf{{#1}}}}
    \newcommand{\NormalTok}[1]{{#1}}
    
    % Additional commands for more recent versions of Pandoc
    \newcommand{\ConstantTok}[1]{\textcolor[rgb]{0.53,0.00,0.00}{{#1}}}
    \newcommand{\SpecialCharTok}[1]{\textcolor[rgb]{0.25,0.44,0.63}{{#1}}}
    \newcommand{\VerbatimStringTok}[1]{\textcolor[rgb]{0.25,0.44,0.63}{{#1}}}
    \newcommand{\SpecialStringTok}[1]{\textcolor[rgb]{0.73,0.40,0.53}{{#1}}}
    \newcommand{\ImportTok}[1]{{#1}}
    \newcommand{\DocumentationTok}[1]{\textcolor[rgb]{0.73,0.13,0.13}{\textit{{#1}}}}
    \newcommand{\AnnotationTok}[1]{\textcolor[rgb]{0.38,0.63,0.69}{\textbf{\textit{{#1}}}}}
    \newcommand{\CommentVarTok}[1]{\textcolor[rgb]{0.38,0.63,0.69}{\textbf{\textit{{#1}}}}}
    \newcommand{\VariableTok}[1]{\textcolor[rgb]{0.10,0.09,0.49}{{#1}}}
    \newcommand{\ControlFlowTok}[1]{\textcolor[rgb]{0.00,0.44,0.13}{\textbf{{#1}}}}
    \newcommand{\OperatorTok}[1]{\textcolor[rgb]{0.40,0.40,0.40}{{#1}}}
    \newcommand{\BuiltInTok}[1]{{#1}}
    \newcommand{\ExtensionTok}[1]{{#1}}
    \newcommand{\PreprocessorTok}[1]{\textcolor[rgb]{0.74,0.48,0.00}{{#1}}}
    \newcommand{\AttributeTok}[1]{\textcolor[rgb]{0.49,0.56,0.16}{{#1}}}
    \newcommand{\InformationTok}[1]{\textcolor[rgb]{0.38,0.63,0.69}{\textbf{\textit{{#1}}}}}
    \newcommand{\WarningTok}[1]{\textcolor[rgb]{0.38,0.63,0.69}{\textbf{\textit{{#1}}}}}
    
    
    % Define a nice break command that doesn't care if a line doesn't already
    % exist.
    \def\br{\hspace*{\fill} \\* }
    % Math Jax compatibility definitions
    \def\gt{>}
    \def\lt{<}
    \let\Oldtex\TeX
    \let\Oldlatex\LaTeX
    \renewcommand{\TeX}{\textrm{\Oldtex}}
    \renewcommand{\LaTeX}{\textrm{\Oldlatex}}
    % Document parameters
    % Document title
    \title{Solution5}
    
    
    
    
    

    % Pygments definitions
    
\makeatletter
\def\PY@reset{\let\PY@it=\relax \let\PY@bf=\relax%
    \let\PY@ul=\relax \let\PY@tc=\relax%
    \let\PY@bc=\relax \let\PY@ff=\relax}
\def\PY@tok#1{\csname PY@tok@#1\endcsname}
\def\PY@toks#1+{\ifx\relax#1\empty\else%
    \PY@tok{#1}\expandafter\PY@toks\fi}
\def\PY@do#1{\PY@bc{\PY@tc{\PY@ul{%
    \PY@it{\PY@bf{\PY@ff{#1}}}}}}}
\def\PY#1#2{\PY@reset\PY@toks#1+\relax+\PY@do{#2}}

\expandafter\def\csname PY@tok@w\endcsname{\def\PY@tc##1{\textcolor[rgb]{0.73,0.73,0.73}{##1}}}
\expandafter\def\csname PY@tok@c\endcsname{\let\PY@it=\textit\def\PY@tc##1{\textcolor[rgb]{0.25,0.50,0.50}{##1}}}
\expandafter\def\csname PY@tok@cp\endcsname{\def\PY@tc##1{\textcolor[rgb]{0.74,0.48,0.00}{##1}}}
\expandafter\def\csname PY@tok@k\endcsname{\let\PY@bf=\textbf\def\PY@tc##1{\textcolor[rgb]{0.00,0.50,0.00}{##1}}}
\expandafter\def\csname PY@tok@kp\endcsname{\def\PY@tc##1{\textcolor[rgb]{0.00,0.50,0.00}{##1}}}
\expandafter\def\csname PY@tok@kt\endcsname{\def\PY@tc##1{\textcolor[rgb]{0.69,0.00,0.25}{##1}}}
\expandafter\def\csname PY@tok@o\endcsname{\def\PY@tc##1{\textcolor[rgb]{0.40,0.40,0.40}{##1}}}
\expandafter\def\csname PY@tok@ow\endcsname{\let\PY@bf=\textbf\def\PY@tc##1{\textcolor[rgb]{0.67,0.13,1.00}{##1}}}
\expandafter\def\csname PY@tok@nb\endcsname{\def\PY@tc##1{\textcolor[rgb]{0.00,0.50,0.00}{##1}}}
\expandafter\def\csname PY@tok@nf\endcsname{\def\PY@tc##1{\textcolor[rgb]{0.00,0.00,1.00}{##1}}}
\expandafter\def\csname PY@tok@nc\endcsname{\let\PY@bf=\textbf\def\PY@tc##1{\textcolor[rgb]{0.00,0.00,1.00}{##1}}}
\expandafter\def\csname PY@tok@nn\endcsname{\let\PY@bf=\textbf\def\PY@tc##1{\textcolor[rgb]{0.00,0.00,1.00}{##1}}}
\expandafter\def\csname PY@tok@ne\endcsname{\let\PY@bf=\textbf\def\PY@tc##1{\textcolor[rgb]{0.82,0.25,0.23}{##1}}}
\expandafter\def\csname PY@tok@nv\endcsname{\def\PY@tc##1{\textcolor[rgb]{0.10,0.09,0.49}{##1}}}
\expandafter\def\csname PY@tok@no\endcsname{\def\PY@tc##1{\textcolor[rgb]{0.53,0.00,0.00}{##1}}}
\expandafter\def\csname PY@tok@nl\endcsname{\def\PY@tc##1{\textcolor[rgb]{0.63,0.63,0.00}{##1}}}
\expandafter\def\csname PY@tok@ni\endcsname{\let\PY@bf=\textbf\def\PY@tc##1{\textcolor[rgb]{0.60,0.60,0.60}{##1}}}
\expandafter\def\csname PY@tok@na\endcsname{\def\PY@tc##1{\textcolor[rgb]{0.49,0.56,0.16}{##1}}}
\expandafter\def\csname PY@tok@nt\endcsname{\let\PY@bf=\textbf\def\PY@tc##1{\textcolor[rgb]{0.00,0.50,0.00}{##1}}}
\expandafter\def\csname PY@tok@nd\endcsname{\def\PY@tc##1{\textcolor[rgb]{0.67,0.13,1.00}{##1}}}
\expandafter\def\csname PY@tok@s\endcsname{\def\PY@tc##1{\textcolor[rgb]{0.73,0.13,0.13}{##1}}}
\expandafter\def\csname PY@tok@sd\endcsname{\let\PY@it=\textit\def\PY@tc##1{\textcolor[rgb]{0.73,0.13,0.13}{##1}}}
\expandafter\def\csname PY@tok@si\endcsname{\let\PY@bf=\textbf\def\PY@tc##1{\textcolor[rgb]{0.73,0.40,0.53}{##1}}}
\expandafter\def\csname PY@tok@se\endcsname{\let\PY@bf=\textbf\def\PY@tc##1{\textcolor[rgb]{0.73,0.40,0.13}{##1}}}
\expandafter\def\csname PY@tok@sr\endcsname{\def\PY@tc##1{\textcolor[rgb]{0.73,0.40,0.53}{##1}}}
\expandafter\def\csname PY@tok@ss\endcsname{\def\PY@tc##1{\textcolor[rgb]{0.10,0.09,0.49}{##1}}}
\expandafter\def\csname PY@tok@sx\endcsname{\def\PY@tc##1{\textcolor[rgb]{0.00,0.50,0.00}{##1}}}
\expandafter\def\csname PY@tok@m\endcsname{\def\PY@tc##1{\textcolor[rgb]{0.40,0.40,0.40}{##1}}}
\expandafter\def\csname PY@tok@gh\endcsname{\let\PY@bf=\textbf\def\PY@tc##1{\textcolor[rgb]{0.00,0.00,0.50}{##1}}}
\expandafter\def\csname PY@tok@gu\endcsname{\let\PY@bf=\textbf\def\PY@tc##1{\textcolor[rgb]{0.50,0.00,0.50}{##1}}}
\expandafter\def\csname PY@tok@gd\endcsname{\def\PY@tc##1{\textcolor[rgb]{0.63,0.00,0.00}{##1}}}
\expandafter\def\csname PY@tok@gi\endcsname{\def\PY@tc##1{\textcolor[rgb]{0.00,0.63,0.00}{##1}}}
\expandafter\def\csname PY@tok@gr\endcsname{\def\PY@tc##1{\textcolor[rgb]{1.00,0.00,0.00}{##1}}}
\expandafter\def\csname PY@tok@ge\endcsname{\let\PY@it=\textit}
\expandafter\def\csname PY@tok@gs\endcsname{\let\PY@bf=\textbf}
\expandafter\def\csname PY@tok@gp\endcsname{\let\PY@bf=\textbf\def\PY@tc##1{\textcolor[rgb]{0.00,0.00,0.50}{##1}}}
\expandafter\def\csname PY@tok@go\endcsname{\def\PY@tc##1{\textcolor[rgb]{0.53,0.53,0.53}{##1}}}
\expandafter\def\csname PY@tok@gt\endcsname{\def\PY@tc##1{\textcolor[rgb]{0.00,0.27,0.87}{##1}}}
\expandafter\def\csname PY@tok@err\endcsname{\def\PY@bc##1{\setlength{\fboxsep}{0pt}\fcolorbox[rgb]{1.00,0.00,0.00}{1,1,1}{\strut ##1}}}
\expandafter\def\csname PY@tok@kc\endcsname{\let\PY@bf=\textbf\def\PY@tc##1{\textcolor[rgb]{0.00,0.50,0.00}{##1}}}
\expandafter\def\csname PY@tok@kd\endcsname{\let\PY@bf=\textbf\def\PY@tc##1{\textcolor[rgb]{0.00,0.50,0.00}{##1}}}
\expandafter\def\csname PY@tok@kn\endcsname{\let\PY@bf=\textbf\def\PY@tc##1{\textcolor[rgb]{0.00,0.50,0.00}{##1}}}
\expandafter\def\csname PY@tok@kr\endcsname{\let\PY@bf=\textbf\def\PY@tc##1{\textcolor[rgb]{0.00,0.50,0.00}{##1}}}
\expandafter\def\csname PY@tok@bp\endcsname{\def\PY@tc##1{\textcolor[rgb]{0.00,0.50,0.00}{##1}}}
\expandafter\def\csname PY@tok@fm\endcsname{\def\PY@tc##1{\textcolor[rgb]{0.00,0.00,1.00}{##1}}}
\expandafter\def\csname PY@tok@vc\endcsname{\def\PY@tc##1{\textcolor[rgb]{0.10,0.09,0.49}{##1}}}
\expandafter\def\csname PY@tok@vg\endcsname{\def\PY@tc##1{\textcolor[rgb]{0.10,0.09,0.49}{##1}}}
\expandafter\def\csname PY@tok@vi\endcsname{\def\PY@tc##1{\textcolor[rgb]{0.10,0.09,0.49}{##1}}}
\expandafter\def\csname PY@tok@vm\endcsname{\def\PY@tc##1{\textcolor[rgb]{0.10,0.09,0.49}{##1}}}
\expandafter\def\csname PY@tok@sa\endcsname{\def\PY@tc##1{\textcolor[rgb]{0.73,0.13,0.13}{##1}}}
\expandafter\def\csname PY@tok@sb\endcsname{\def\PY@tc##1{\textcolor[rgb]{0.73,0.13,0.13}{##1}}}
\expandafter\def\csname PY@tok@sc\endcsname{\def\PY@tc##1{\textcolor[rgb]{0.73,0.13,0.13}{##1}}}
\expandafter\def\csname PY@tok@dl\endcsname{\def\PY@tc##1{\textcolor[rgb]{0.73,0.13,0.13}{##1}}}
\expandafter\def\csname PY@tok@s2\endcsname{\def\PY@tc##1{\textcolor[rgb]{0.73,0.13,0.13}{##1}}}
\expandafter\def\csname PY@tok@sh\endcsname{\def\PY@tc##1{\textcolor[rgb]{0.73,0.13,0.13}{##1}}}
\expandafter\def\csname PY@tok@s1\endcsname{\def\PY@tc##1{\textcolor[rgb]{0.73,0.13,0.13}{##1}}}
\expandafter\def\csname PY@tok@mb\endcsname{\def\PY@tc##1{\textcolor[rgb]{0.40,0.40,0.40}{##1}}}
\expandafter\def\csname PY@tok@mf\endcsname{\def\PY@tc##1{\textcolor[rgb]{0.40,0.40,0.40}{##1}}}
\expandafter\def\csname PY@tok@mh\endcsname{\def\PY@tc##1{\textcolor[rgb]{0.40,0.40,0.40}{##1}}}
\expandafter\def\csname PY@tok@mi\endcsname{\def\PY@tc##1{\textcolor[rgb]{0.40,0.40,0.40}{##1}}}
\expandafter\def\csname PY@tok@il\endcsname{\def\PY@tc##1{\textcolor[rgb]{0.40,0.40,0.40}{##1}}}
\expandafter\def\csname PY@tok@mo\endcsname{\def\PY@tc##1{\textcolor[rgb]{0.40,0.40,0.40}{##1}}}
\expandafter\def\csname PY@tok@ch\endcsname{\let\PY@it=\textit\def\PY@tc##1{\textcolor[rgb]{0.25,0.50,0.50}{##1}}}
\expandafter\def\csname PY@tok@cm\endcsname{\let\PY@it=\textit\def\PY@tc##1{\textcolor[rgb]{0.25,0.50,0.50}{##1}}}
\expandafter\def\csname PY@tok@cpf\endcsname{\let\PY@it=\textit\def\PY@tc##1{\textcolor[rgb]{0.25,0.50,0.50}{##1}}}
\expandafter\def\csname PY@tok@c1\endcsname{\let\PY@it=\textit\def\PY@tc##1{\textcolor[rgb]{0.25,0.50,0.50}{##1}}}
\expandafter\def\csname PY@tok@cs\endcsname{\let\PY@it=\textit\def\PY@tc##1{\textcolor[rgb]{0.25,0.50,0.50}{##1}}}

\def\PYZbs{\char`\\}
\def\PYZus{\char`\_}
\def\PYZob{\char`\{}
\def\PYZcb{\char`\}}
\def\PYZca{\char`\^}
\def\PYZam{\char`\&}
\def\PYZlt{\char`\<}
\def\PYZgt{\char`\>}
\def\PYZsh{\char`\#}
\def\PYZpc{\char`\%}
\def\PYZdl{\char`\$}
\def\PYZhy{\char`\-}
\def\PYZsq{\char`\'}
\def\PYZdq{\char`\"}
\def\PYZti{\char`\~}
% for compatibility with earlier versions
\def\PYZat{@}
\def\PYZlb{[}
\def\PYZrb{]}
\makeatother


    % Exact colors from NB
    \definecolor{incolor}{rgb}{0.0, 0.0, 0.5}
    \definecolor{outcolor}{rgb}{0.545, 0.0, 0.0}



    
    % Prevent overflowing lines due to hard-to-break entities
    \sloppy 
    % Setup hyperref package
    \hypersetup{
      breaklinks=true,  % so long urls are correctly broken across lines
      colorlinks=true,
      urlcolor=urlcolor,
      linkcolor=linkcolor,
      citecolor=citecolor,
      }
    % Slightly bigger margins than the latex defaults
    
    \geometry{verbose,tmargin=1in,bmargin=1in,lmargin=1in,rmargin=1in}
    
    
\usepackage{graphicx}
\usepackage{amsmath}

%\setlength{\parindent}{2em}
\setlength{\parskip}{0ex}
\setlength{\oddsidemargin}{-.3in}
\setlength{\textwidth}{6.5in}
\setlength{\topmargin}{-.5in}
\setlength{\textheight}{9in}

\newcommand{\comment}[1]{}
\renewcommand{\thepage}{}
    \begin{document}
    
  
%\renewcommand{\baselinestretch}{.95}

\setcounter{page}{1}
\thispagestyle{empty}

\begin{center}
\LARGE
CSE 321: Algorithms \\
Solutions of Homework 3\\
\normalsize 
\ \\
Fall 2018 \\
Perihan Mirkelam, Scientific Preparation

\end{center}

\normalsize

\begin{enumerate}

\item ?

\pagebreak
% Solution 2
\item 
$n$: Number of chips. \\
$m$: Maximum number of taken chips. \\
The function “startGame” is calling recursively by getting three parameters as n, m and which player's turn is. \\
startGame(n, m, True) \\

Ask user an input number between $1-m$. Each time the function calls itself by decreasing $n$ according to user input. This recursive call continiuous until $m-1$ divides remaining chips. Function returns which player's turn as winner at this base condition. \\
startGame(n - input, m, not isFirstUser) \\


Best case is to win in first move: This takes constant time. \\
Worst case is $n+1$ divides $m$ and each player takes 1 chip over $m$ moves. Then $n-m$ divides $n$: This takes $m$ times. \\
Time complexity is linear: $\mathcal{O}(m)$ \\

\begin{Verbatim}[commandchars=\\\{\}]
{\color{incolor}In [{\color{incolor} }]:} \PY{n}{n} \PY{o}{=} \PY{l+m+mi}{19} \PY{c+c1}{\PYZsh{}Number of chips}
        \PY{n}{m} \PY{o}{=} \PY{l+m+mi}{3}  \PY{c+c1}{\PYZsh{}Number of maximum taken chips}
        
        
        \PY{k}{def} \PY{n+nf}{startGame}\PY{p}{(}\PY{n}{n}\PY{p}{,} \PY{n}{m}\PY{p}{,} \PY{n}{isFirstPlayer}\PY{p}{)}\PY{p}{:}
            \PY{n+nb}{print}\PY{p}{(}\PY{l+s+s2}{\PYZdq{}}\PY{l+s+s2}{n }\PY{l+s+s2}{\PYZdq{}}\PY{p}{,} \PY{n}{n}\PY{p}{,} \PY{l+s+s2}{\PYZdq{}}\PY{l+s+s2}{, m}\PY{l+s+s2}{\PYZdq{}}\PY{p}{,} \PY{n}{m}\PY{p}{)}
            \PY{k}{if} \PY{n}{isFirstPlayer} \PY{p}{:}
                \PY{n}{player} \PY{o}{=} \PY{l+s+s1}{\PYZsq{}}\PY{l+s+s1}{First}\PY{l+s+s1}{\PYZsq{}}
            \PY{k}{else}\PY{p}{:}
                \PY{n}{player} \PY{o}{=} \PY{l+s+s1}{\PYZsq{}}\PY{l+s+s1}{Second}\PY{l+s+s1}{\PYZsq{}}
                
            \PY{k}{if} \PY{o+ow}{not} \PY{n}{n} \PY{o}{\PYZpc{}} \PY{p}{(}\PY{n}{m}\PY{o}{+}\PY{l+m+mi}{1}\PY{p}{)} \PY{o}{==} \PY{l+m+mi}{0}\PY{p}{:}
                \PY{n}{ask} \PY{o}{=} \PY{n+nb}{str}\PY{p}{(}\PY{n}{player}\PY{p}{)} \PY{o}{+} \PY{l+s+s1}{\PYZsq{}}\PY{l+s+s1}{ player. How many chips will you get: }\PY{l+s+s1}{\PYZsq{}}
                \PY{n}{var} \PY{o}{=} \PY{n+nb}{int}\PY{p}{(}\PY{n+nb}{input}\PY{p}{(}\PY{n}{ask}\PY{p}{)}\PY{p}{)}
                \PY{k}{if} \PY{n}{var} \PY{o}{\PYZgt{}} \PY{l+m+mi}{3} \PY{o+ow}{or} \PY{n}{var} \PY{o}{\PYZlt{}} \PY{l+m+mi}{1} \PY{p}{:}
                    \PY{n}{error} \PY{o}{=} \PY{l+s+s2}{\PYZdq{}}\PY{l+s+s2}{You can get at least 1 and at most }\PY{l+s+s2}{\PYZdq{}}\PY{o}{+} \PY{n+nb}{str}\PY{p}{(}\PY{n}{m}\PY{p}{)} \PY{o}{+} \PY{l+s+s2}{\PYZdq{}}\PY{l+s+s2}{ chips.}\PY{l+s+s2}{\PYZdq{}}
                    \PY{n+nb}{print}\PY{p}{(}\PY{n}{error}\PY{p}{)}
                    \PY{n}{var} \PY{o}{=} \PY{n+nb}{int}\PY{p}{(}\PY{n+nb}{input}\PY{p}{(}\PY{n}{ask}\PY{p}{)}\PY{p}{)}
                \PY{n}{startGame}\PY{p}{(}\PY{n}{n}\PY{o}{\PYZhy{}}\PY{n}{var}\PY{p}{,} \PY{n}{m}\PY{p}{,} \PY{o+ow}{not} \PY{n}{isFirstPlayer}\PY{p}{)} 
            \PY{k}{return} \PY{n}{player}
            
        \PY{n+nb}{print}\PY{p}{(}\PY{n}{startGame}\PY{p}{(}\PY{n}{n}\PY{p}{,} \PY{n}{m}\PY{p}{,} \PY{k+kc}{True}\PY{p}{)}\PY{p}{,} \PY{l+s+s2}{\PYZdq{}}\PY{l+s+s2}{player wins!}\PY{l+s+s2}{\PYZdq{}}\PY{p}{)}
\end{Verbatim}

\pagebreak
% Solution 3
\item  Method "search" uses binary search algorithm. Since the array is already sorted and has increasing integers, not search for unnecessary nodes by controlling whether $A[i]$ is greater or smaller than $i$.

Since we are using binary search, for each call to search the difference between upperBound and lowerBound is halved. Hence, the running time is: $\mathcal{O}(\log n)$

 \begin{Verbatim}[commandchars=\\\{\}]
{\color{incolor}In [{\color{incolor}}]:} \PY{k+kn}{import} \PY{n+nn}{array} \PY{k}{as} \PY{n+nn}{arr}
         
         \PY{k}{def} \PY{n+nf}{search}\PY{p}{(}\PY{n}{A}\PY{p}{,}\PY{n}{lowerBound}\PY{p}{,}\PY{n}{upperBound}\PY{p}{)}\PY{p}{:} 
             \PY{n}{mid} \PY{o}{=} \PY{n+nb}{int}\PY{p}{(}\PY{p}{(}\PY{n}{upperBound} \PY{o}{+} \PY{n}{lowerBound}\PY{p}{)}\PY{o}{/}\PY{l+m+mi}{2}\PY{p}{)} 
             \PY{n+nb}{print}\PY{p}{(}\PY{l+s+s2}{\PYZdq{}}\PY{l+s+s2}{mid: }\PY{l+s+s2}{\PYZdq{}}\PY{p}{,} \PY{n}{mid}\PY{p}{,} \PY{l+s+s2}{\PYZdq{}}\PY{l+s+s2}{ A[mid]: }\PY{l+s+s2}{\PYZdq{}}\PY{p}{,} \PY{n}{A}\PY{p}{[}\PY{n}{mid}\PY{p}{]}\PY{p}{,} \PY{l+s+s2}{\PYZdq{}}\PY{l+s+s2}{ lowerBound: }\PY{l+s+s2}{\PYZdq{}}\PY{p}{,} \PY{n}{lowerBound}\PY{p}{,} \PY{l+s+s2}{\PYZdq{}}\PY{l+s+s2}{ upperBound: }\PY{l+s+s2}{\PYZdq{}}\PY{p}{,} \PY{n}{upperBound}\PY{p}{)} 
             \PY{k}{if} \PY{n}{lowerBound} \PY{o}{==} \PY{n}{upperBound} \PY{o+ow}{and} \PY{n}{A}\PY{p}{[}\PY{n}{mid}\PY{p}{]} \PY{o}{!=} \PY{n}{mid} \PY{p}{:} 
                 \PY{n+nb}{print}\PY{p}{(}\PY{l+s+s2}{\PYZdq{}}\PY{l+s+s2}{NO. There is no element on A array such that A[i] = i.}\PY{l+s+s2}{\PYZdq{}}\PY{p}{)}
             \PY{k}{if} \PY{n}{A}\PY{p}{[}\PY{n}{mid}\PY{p}{]} \PY{o}{==} \PY{n}{mid}\PY{p}{:}
                 \PY{n+nb}{print}\PY{p}{(}\PY{l+s+s2}{\PYZdq{}}\PY{l+s+s2}{There exist an element (}\PY{l+s+s2}{\PYZdq{}}\PY{p}{,} \PY{n}{mid}\PY{p}{,} \PY{l+s+s2}{\PYZdq{}}\PY{l+s+s2}{th element) of array A such that A[i] = i =}\PY{l+s+s2}{\PYZdq{}}\PY{p}{,} \PY{n}{mid}\PY{p}{)} 
             \PY{k}{if} \PY{n}{A}\PY{p}{[}\PY{n}{mid}\PY{p}{]} \PY{o}{\PYZgt{}} \PY{n}{mid} \PY{p}{:} 
                 \PY{n}{search}\PY{p}{(}\PY{n}{A}\PY{p}{,}\PY{n}{lowerBound}\PY{p}{,}\PY{n}{mid}\PY{p}{)} 
             \PY{k}{if} \PY{n}{A}\PY{p}{[}\PY{n}{mid}\PY{p}{]} \PY{o}{\PYZlt{}} \PY{n}{mid} \PY{p}{:}
                 \PY{n}{search}\PY{p}{(}\PY{n}{A}\PY{p}{,}\PY{n}{mid} \PY{o}{+} \PY{l+m+mi}{1}\PY{p}{,}\PY{n}{upperBound}\PY{p}{)}
         
         \PY{n}{A} \PY{o}{=} \PY{n}{arr}\PY{o}{.}\PY{n}{array}\PY{p}{(}\PY{l+s+s2}{\PYZdq{}}\PY{l+s+s2}{i}\PY{l+s+s2}{\PYZdq{}}\PY{p}{,} \PY{p}{[}\PY{o}{\PYZhy{}}\PY{l+m+mi}{4}\PY{p}{,} \PY{o}{\PYZhy{}}\PY{l+m+mi}{2}\PY{p}{,} \PY{o}{\PYZhy{}}\PY{l+m+mi}{1}\PY{p}{,} \PY{l+m+mi}{0}\PY{p}{,} \PY{l+m+mi}{4}\PY{p}{,} \PY{l+m+mi}{6}\PY{p}{,} \PY{l+m+mi}{7}\PY{p}{]}\PY{p}{)}
         \PY{n}{search}\PY{p}{(}\PY{n}{A}\PY{p}{,} \PY{l+m+mi}{0}\PY{p}{,} \PY{n+nb}{len}\PY{p}{(}\PY{n}{A}\PY{p}{)}\PY{p}{)}\PY{p}{;}
\end{Verbatim}

    \begin{Verbatim}[commandchars=\\\{\}]
mid:  3  A[mid]:  0  lowerBound:  0  upperBound:  7
mid:  5  A[mid]:  6  lowerBound:  4  upperBound:  7
mid:  4  A[mid]:  4  lowerBound:  4  upperBound:  5
There exist an element ( 4 th element) of array A such that A[i] = i = 4

    \end{Verbatim}

\pagebreak
%Solution 4
\item
Method "calcMaxSubArraySum" is calling recursively each time divided by $2$ for two halves. Therefore, recurrence relation will be, \\
$T(n) = 2T(n/2) + f(n)$\\
By Master Theorem, time complexity of our algorithm: $\Theta(n \log n)$

  \begin{Verbatim}[commandchars=\\\{\}]
{\color{incolor}In [{\color{incolor}}]:} \PY{k+kn}{import} \PY{n+nn}{sys}
           
         \PY{c+c1}{\PYZsh{} Calculated max sum is maximum of three cases. }
         \PY{c+c1}{\PYZsh{} 1. Calculate right half and left half recursively.}
         \PY{c+c1}{\PYZsh{} 2. Plus calculate sub array(first time whole array) }
         \PY{c+c1}{\PYZsh{} include middle point and its both side of sum }
         \PY{k}{def} \PY{n+nf}{calcMaxSubArraySum}\PY{p}{(}\PY{n}{arr}\PY{p}{,} \PY{n}{left}\PY{p}{,} \PY{n}{right}\PY{p}{)} \PY{p}{:} 
         
             \PY{k}{if} \PY{p}{(}\PY{n}{left} \PY{o}{==} \PY{n}{right}\PY{p}{)} \PY{p}{:} 
                 \PY{k}{return} \PY{n}{arr}\PY{p}{[}\PY{n}{left}\PY{p}{]} 
           
             \PY{n}{mid} \PY{o}{=} \PY{n+nb}{int}\PY{p}{(}\PY{p}{(}\PY{n}{left} \PY{o}{+} \PY{n}{right}\PY{p}{)} \PY{o}{/} \PY{l+m+mi}{2}\PY{p}{)}
             
             \PY{n}{leftside} \PY{o}{=} \PY{n}{calcMaxSubArraySum}\PY{p}{(}\PY{n}{arr}\PY{p}{,} \PY{n}{left}\PY{p}{,} \PY{n}{mid}\PY{p}{)}
             \PY{n}{rightside} \PY{o}{=} \PY{n}{calcMaxSubArraySum}\PY{p}{(}\PY{n}{arr}\PY{p}{,} \PY{n}{mid}\PY{o}{+}\PY{l+m+mi}{1}\PY{p}{,} \PY{n}{right}\PY{p}{)}
                   
             \PY{c+c1}{\PYZsh{}calculate max sum of left and right side of mid}
             \PY{n}{temp\PYZus{}left} \PY{o}{=} \PY{l+m+mi}{0} 
             \PY{n}{left\PYZus{}sum} \PY{o}{=} \PY{o}{\PYZhy{}}\PY{n}{sys}\PY{o}{.}\PY{n}{maxsize}
               
             \PY{k}{for} \PY{n}{i} \PY{o+ow}{in} \PY{n+nb}{range}\PY{p}{(}\PY{n}{mid}\PY{p}{,} \PY{n}{left}\PY{o}{\PYZhy{}}\PY{l+m+mi}{1}\PY{p}{,} \PY{o}{\PYZhy{}}\PY{l+m+mi}{1}\PY{p}{)} \PY{p}{:} 
                 \PY{n}{temp\PYZus{}left} \PY{o}{=} \PY{n}{temp\PYZus{}left} \PY{o}{+} \PY{n}{arr}\PY{p}{[}\PY{n}{i}\PY{p}{]} 
                 \PY{n}{left\PYZus{}sum} \PY{o}{=} \PY{n+nb}{max}\PY{p}{(}\PY{n}{left\PYZus{}sum}\PY{p}{,} \PY{n}{temp\PYZus{}left}\PY{p}{)}
               
             \PY{n}{temp\PYZus{}right} \PY{o}{=} \PY{l+m+mi}{0}
             \PY{n}{right\PYZus{}sum} \PY{o}{=} \PY{o}{\PYZhy{}}\PY{n}{sys}\PY{o}{.}\PY{n}{maxsize}
             \PY{k}{for} \PY{n}{i} \PY{o+ow}{in} \PY{n+nb}{range}\PY{p}{(}\PY{n}{mid} \PY{o}{+} \PY{l+m+mi}{1}\PY{p}{,} \PY{n}{right} \PY{o}{+} \PY{l+m+mi}{1}\PY{p}{)} \PY{p}{:} 
                 \PY{n}{temp\PYZus{}right} \PY{o}{=} \PY{n}{temp\PYZus{}right} \PY{o}{+} \PY{n}{arr}\PY{p}{[}\PY{n}{i}\PY{p}{]}  
                 \PY{n}{right\PYZus{}sum} \PY{o}{=} \PY{n+nb}{max}\PY{p}{(}\PY{n}{right\PYZus{}sum}\PY{p}{,} \PY{n}{temp\PYZus{}right}\PY{p}{)}
               
             \PY{n}{centered} \PY{o}{=} \PY{n}{left\PYZus{}sum} \PY{o}{+} \PY{n}{right\PYZus{}sum}\PY{p}{;}
           
             \PY{k}{return} \PY{n+nb}{max}\PY{p}{(}\PY{n}{leftside}\PY{p}{,} \PY{n}{rightside}\PY{p}{,} \PY{n}{centered}\PY{p}{)}               
           
         \PY{n}{A} \PY{o}{=} \PY{p}{[}\PY{l+m+mi}{5}\PY{p}{,} \PY{o}{\PYZhy{}}\PY{l+m+mi}{6}\PY{p}{,} \PY{l+m+mi}{6}\PY{p}{,} \PY{l+m+mi}{7}\PY{p}{,} \PY{o}{\PYZhy{}}\PY{l+m+mi}{6}\PY{p}{,} \PY{l+m+mi}{7}\PY{p}{,} \PY{o}{\PYZhy{}}\PY{l+m+mi}{4}\PY{p}{,} \PY{l+m+mi}{3}\PY{p}{]}   
         \PY{n}{maxSum} \PY{o}{=} \PY{n}{calcMaxSubArraySum}\PY{p}{(}\PY{n}{A}\PY{p}{,} \PY{l+m+mi}{0}\PY{p}{,} \PY{n+nb}{len}\PY{p}{(}\PY{n}{arr}\PY{p}{)}\PY{o}{\PYZhy{}}\PY{l+m+mi}{1}\PY{p}{)} 
         \PY{n+nb}{print}\PY{p}{(}\PY{l+s+s2}{\PYZdq{}}\PY{l+s+s2}{Max sum of subset is: }\PY{l+s+s2}{\PYZdq{}}\PY{p}{,} \PY{n}{maxSum}\PY{p}{)} 
\end{Verbatim}

    \begin{Verbatim}[commandchars=\\\{\}]
Max sum of subset is:  14

    \end{Verbatim}

\pagebreak
% Solution 5
\item The function “match” is calling recursively by getting two parameters “text” as a string and “pattern” as a string list. \\

When the function is called first, the pattern will have a pattern list with at least one element: \\
match('Tobeornottobe', [A, B, C, D, A, B, C]) \\

Then the function will be called recursively in a loop so that the "pattern" list with an exact one element and part of the "text": \\
match('be', [B]) \\


Let $n$: Length of “text” string. \\
Let $p$: Number of pattern list items. \\
Let $m_i$: Size of $i$th string item of pattern list. \\
Let $M$: Total count of “pattern” list items. $$M = \sum_{i=0}^{p-1} m_i$$ \\
Time complexity is linear: $\mathcal{O}(M) = \mathcal{O}(n)$ \\
    
\begin{Verbatim}[commandchars=\\\{\}]
 {\color{incolor}In [{\color{incolor} }]:} \PY{n}{text} \PY{o}{=} \PY{l+s+s1}{\PYZsq{}}\PY{l+s+s1}{Tobeornottobe}\PY{l+s+s1}{\PYZsq{}}
         \PY{n}{A} \PY{o}{=} \PY{l+s+s1}{\PYZsq{}}\PY{l+s+s1}{to}\PY{l+s+s1}{\PYZsq{}}
         \PY{n}{B} \PY{o}{=} \PY{l+s+s1}{\PYZsq{}}\PY{l+s+s1}{be}\PY{l+s+s1}{\PYZsq{}}
         \PY{n}{C} \PY{o}{=} \PY{l+s+s1}{\PYZsq{}}\PY{l+s+s1}{or}\PY{l+s+s1}{\PYZsq{}}
         \PY{n}{D} \PY{o}{=} \PY{l+s+s1}{\PYZsq{}}\PY{l+s+s1}{not}\PY{l+s+s1}{\PYZsq{}}
         \PY{n}{pattern} \PY{o}{=} \PY{p}{[}\PY{n}{A}\PY{p}{,} \PY{n}{B}\PY{p}{,} \PY{n}{C}\PY{p}{,} \PY{n}{D}\PY{p}{,} \PY{n}{A}\PY{p}{,} \PY{n}{B}\PY{p}{]}
         
         \PY{k}{def} \PY{n+nf}{match}\PY{p}{(}\PY{n}{text}\PY{p}{,} \PY{n}{pattern}\PY{p}{)}\PY{p}{:}
             \PY{n}{textLength} \PY{o}{=} \PY{n+nb}{len}\PY{p}{(}\PY{n}{text}\PY{p}{)}
             \PY{n}{patternCharLength} \PY{o}{=} \PY{l+m+mi}{0}
             
             \PY{k}{for} \PY{n}{pat} \PY{o+ow}{in} \PY{n}{pattern}\PY{p}{:}
                 \PY{n}{patternCharLength} \PY{o}{+}\PY{o}{=} \PY{n+nb}{len}\PY{p}{(}\PY{n}{pat}\PY{p}{)}
             
             \PY{c+c1}{\PYZsh{}text length and total length of pattern items must match}
             \PY{k}{if} \PY{n+nb}{len}\PY{p}{(}\PY{n}{pattern}\PY{p}{)} \PY{o}{==} \PY{l+m+mi}{1} \PY{o+ow}{or} \PY{n}{patternCharLength} \PY{o}{==} \PY{n}{textLength}\PY{p}{:}
                 \PY{n}{startIndex} \PY{o}{=} \PY{l+m+mi}{0}
                 \PY{n}{endIndex} \PY{o}{=} \PY{l+m+mi}{0}
                 \PY{k}{if} \PY{n+nb}{len}\PY{p}{(}\PY{n}{pattern}\PY{p}{)} \PY{o}{\PYZgt{}} \PY{l+m+mi}{1}\PY{p}{:}
                     \PY{c+c1}{\PYZsh{}this part works when the function is called for the first time}
                     \PY{c+c1}{\PYZsh{}the recursive call for each pattern element}
                     \PY{k}{for} \PY{n}{i} \PY{o+ow}{in} \PY{n+nb}{range}\PY{p}{(}\PY{n+nb}{len}\PY{p}{(}\PY{n}{pattern}\PY{p}{)}\PY{p}{)}\PY{p}{:}
                         \PY{n}{partOfPattern} \PY{o}{=} \PY{p}{[}\PY{n}{pattern}\PY{p}{[}\PY{n}{i}\PY{p}{]}\PY{p}{]}
                         \PY{n}{endIndex} \PY{o}{=} \PY{n}{startIndex} \PY{o}{+} \PY{n+nb}{len}\PY{p}{(}\PY{n}{pattern}\PY{p}{[}\PY{n}{i}\PY{p}{]}\PY{p}{)}
                         \PY{k}{if} \PY{o+ow}{not} \PY{n}{match}\PY{p}{(}\PY{n}{text}\PY{p}{[}\PY{n}{startIndex}\PY{p}{:} \PY{n}{endIndex}\PY{p}{]}\PY{p}{,} \PY{n}{partOfPattern}\PY{p}{)}\PY{p}{:}
                             \PY{n+nb}{print}\PY{p}{(}\PY{l+s+s2}{\PYZdq{}}\PY{l+s+s2}{Not match text part: }\PY{l+s+s2}{\PYZdq{}}\PY{p}{,} \PY{n}{text}\PY{p}{[}\PY{n}{startIndex}\PY{p}{:}\PY{n}{endIndex}\PY{p}{]}\PY{p}{,} \PY{l+s+s2}{\PYZdq{}}\PY{l+s+s2}{ partOfPattern: }\PY{l+s+s2}{\PYZdq{}}\PY{p}{,} \PY{n}{partOfPattern}\PY{p}{)}
                             \PY{k}{return} \PY{k+kc}{False}
                         \PY{n}{startIndex} \PY{o}{=} \PY{n}{endIndex}
                         
                 \PY{c+c1}{\PYZsh{}this part works for each element of pattern list after the function \\ called for the first time}
                 \PY{c+c1}{\PYZsh{}complexity is m\PYZus{}i}
                 \PY{k}{if} \PY{n+nb}{len}\PY{p}{(}\PY{n}{pattern}\PY{p}{)} \PY{o}{==} \PY{l+m+mi}{1}\PY{p}{:}
                     \PY{n}{patternItem} \PY{o}{=} \PY{n}{pattern}\PY{p}{[}\PY{l+m+mi}{0}\PY{p}{]}
                     \PY{k}{for} \PY{n}{i} \PY{o+ow}{in} \PY{n+nb}{range}\PY{p}{(}\PY{n+nb}{len}\PY{p}{(}\PY{n}{patternItem}\PY{p}{)}\PY{p}{)}\PY{p}{:}
                         \PY{k}{if} \PY{o+ow}{not} \PY{n}{patternItem}\PY{p}{[}\PY{n}{i}\PY{p}{]}\PY{o}{.}\PY{n}{lower}\PY{p}{(}\PY{p}{)} \PY{o}{==} \PY{n}{text}\PY{p}{[}\PY{n}{i}\PY{p}{]}\PY{o}{.}\PY{n}{lower}\PY{p}{(}\PY{p}{)}\PY{p}{:}
                             \PY{k}{return} \PY{k+kc}{False}
                    \PY{c+c1}{\PYZsh{} print(patternItem, \PYZdq{} matches to \PYZdq{}, text)}
                 \PY{k}{return} \PY{k+kc}{True}
             \PY{k}{else}\PY{p}{:}
                 \PY{n+nb}{print}\PY{p}{(}\PY{l+s+s2}{\PYZdq{}}\PY{l+s+s2}{Not valid sizes. Pattern sizes are greater than text size!}\PY{l+s+s2}{\PYZdq{}}\PY{p}{)}
                 \PY{k}{return} \PY{k+kc}{False}
         
         \PY{n+nb}{print}\PY{p}{(}\PY{n}{text}\PY{p}{,} \PY{l+s+s2}{\PYZdq{}}\PY{l+s+s2}{ matches to }\PY{l+s+s2}{\PYZdq{}}\PY{p}{,} \PY{n}{pattern}\PY{p}{,} \PY{l+s+s2}{\PYZdq{}}\PY{l+s+s2}{: }\PY{l+s+s2}{\PYZdq{}}\PY{p}{,} \PY{n}{match}\PY{p}{(}\PY{n}{text}\PY{p}{,} \PY{n}{pattern}\PY{p}{)}\PY{p}{)}
\end{Verbatim}
    
\end{enumerate}    
\end{document}
